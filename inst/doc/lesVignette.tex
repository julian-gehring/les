%% LyX 1.6.5 created this file.  For more info, see http://www.lyx.org/.
%% Do not edit unless you really know what you are doing.
\documentclass[english]{article}
\usepackage[T1]{fontenc}
\usepackage[utf8]{inputenc}
\setlength{\parskip}{\medskipamount}
\setlength{\parindent}{0pt}
\usepackage{babel}

\usepackage[unicode=true, pdfusetitle,
 bookmarks=true,bookmarksnumbered=false,bookmarksopen=false,
 breaklinks=true,pdfborder={0 0 0},backref=false,colorlinks=false]
 {hyperref}

\makeatletter
%%%%%%%%%%%%%%%%%%%%%%%%%%%%%% User specified LaTeX commands.
%\VignetteIndexEntry{Detection of Loci of Enhanced Significance (LES) in tiling array data with the les package}
%\VignettePackage{les}

\usepackage{fancyvrb}
\fvset{listparameters={\setlength{\topsep}{0pt}}}

\newcommand{\Rfunction}[1]{{\texttt{#1}}}
\newcommand{\Robject}[1]{{\texttt{#1}}}
\newcommand{\Rpackage}[1]{{\textit{#1}}}
\newcommand{\Rclass}[1]{{\textit{#1}}}

\makeatother

\usepackage{Sweave}
\begin{document}

\title{Introduction to the les package:\\
Loci of Enhanced Significance in Tiling Array Data}


\author{Julian Gehring}
\maketitle
\begin{abstract}
In this vignette we describe using the les package for finding Loci
of Enhanced Significance (LES) in tiling microarray data. With an
example of a general framework we illustrate how to apply the package
for exploring regions of regulation in differential expression and
chip-CHIP analysis.
\end{abstract}

\section{Introduction}

Tiling microarrays have become an important platform for the investigation
of regulation in expression and DNA-protein interaction. They provide
a relatively unbiased tool covering large regions of interest in the
genome.

Beside the analysis of single microarrays the investigation of differential
effects between experimental conditions is critical for current research.
A common approach consists in applying statistical tests on the level
of single probes and thereby computing p-values for each probe independently.
Since the targets of such experiments cover areas with several probes
the logical next step involves combining information from neighboring
probes into a reasonable statistic. In regions with differential effects
the test statistics change their distribution and are referred to
as Loci of Enhanced Significance (LES). The changes in the test statistics
depend on the underlying test applied.

The les package provides the ability to detect such LES independent
of the underlying statistical test and can therefore be used for a
wide range of applications. This vignette illustrates how to LES can
be found in tiling microarray data sets.



\section{Data and statistics on probe level}

For this analysis we will use a simulated data set describing differential
expression between two conditions. It contains 1000 probes with 3
chips each for the conditions treatment and control. The expression
values are stored in an expression set. We will extract the position
of the probes, the conditions of the samples and the expression values.
There are two regions with changes present in the data, each being
50 bp long.

\begin{Schunk}
\begin{Sinput}
> library(les)
> library(Biobase)
> data(simTile)
> treatment <- as.factor(phenoData(simTile)$condition == 
+     "treatment")